% Metódy inžinierskej práce

\documentclass[10pt,twoside,slovak,a4paper]{article}

\usepackage[slovak]{babel}

\usepackage[IL2]{fontenc} % lepšia sadzba písmena Ľ než v T1
\usepackage[utf8]{inputenc}
\usepackage{graphicx}
\usepackage{url} % príkaz \url na formátovanie URL
\usepackage{hyperref} % odkazy v texte budú aktívne (pri niektorých triedach dokumentov spôsobuje posun textu)

\usepackage{cite}
%\usepackage{times}

\pagestyle{headings}

\title{Detekcia malwaru pomocou hlbokého učenia 
založeného na vízii\thanks{Semestrálny projekt v predmete Metódy inžinierskej práce, ak. rok 2021/22, vedenie: Fedor Lehocki}} % meno a priezvisko vyučujúceho na cvičeniach

\author{Jakub Martinak\\[2pt]
	{\small Slovenská technická univerzita v Bratislave}\\
	{\small Fakulta informatiky a informačných technológií}\\
	{\small \texttt{xmartinakj@stuba.sk}}
	}

\date{\small 11. oktober 2021} % upravte



\begin{document}

\maketitle

\begin{abstract}
Keďže v 21. storočí je čím ďalej tím viac potreba chrániť naše elektronické 
zariadenia pred nebezpečným malwarom tak sa chcem zamerať na túto tému 
v oblasti informačnej bezpečnosti a implementovanie strojového učenia do nej.
V tomto článku na začiatku objasním modelovanie v softwarovom inžinierstve 
a ako to súvisí s mojou témou a neskôr sa budem zameriavať na presnosť 
určenia či sa jedná o malware alebo nie pomocou Deep Learning Vision
a taktiež budem rozoberať problémy a výhody spojené s touto formou 
detekcie, výhody a nevýhody klasickej detekcie malwaru a jeho limity oproti 
strojovému učeniu.
\end{abstract}



\section{Úvod}

Túto tému som si vybral z dôvodu nárastu kybernetických útokov vo svete a následnej obrane voči nim pomocou moderných technológií a využitím umelej inteligencie. V prvej časti článku sa budem venovať modelovanie v softwarovom inžinierstve~\ref{modelovanie}. Druhá časť sa bude zaoberať ~\ref{nedávnemu PROGRESU v kybernetickej bezpečnosti}, z ktorej následne prejdem na hlavnú tému tohto článku ~\ref{Detekcia malwaru pomocou hlbokého učenia 
založeného na vízii}.
Záverečné poznámky prináša časť~\ref{zaver}.



\section{Modelovanie v Softwarovom inžinierstve} \label{modelovanie}

Z obr.~\ref{f:rozhod} je všetko jasné. 

\begin{figure*}[tbh]
\centering
%\includegraphics[scale=1.0]{diagram.pdf}

\end{figure*}



\section{Iná časť} \label{ina}

Základným problémom je teda\ldots{} Najprv sa pozrieme na nejaké vysvetlenie (časť~\ref{ina:nejake}), a potom na ešte nejaké (časť~\ref{ina:nejake}).\footnote{Niekedy môžete potrebovať aj poznámku pod čiarou.}

Môže sa zdať, že problém vlastne nejestvuje\cite{Coplien:MPD}, ale bolo dokázané, že to tak nie je~\cite{Czarnecki:Staged, Czarnecki:Progress}. Napriek tomu, aj dnes na webe narazíme na všelijaké pochybné názory\cite{PLP-Framework}. Dôležité veci možno \emph{zdôrazniť kurzívou}.


\subsection{Nejaké vysvetlenie} \label{ina:nejake}

Niekedy treba uviesť zoznam:

\begin{itemize}
\item jedna vec
\item druhá vec
	\begin{itemize}
	\item x
	\item y
	\end{itemize}
\end{itemize}

Ten istý zoznam, len číslovaný:

\begin{enumerate}
\item jedna vec
\item druhá vec
	\begin{enumerate}
	\item x
	\item y
	\end{enumerate}
\end{enumerate}


\subsection{Ešte nejaké vysvetlenie} \label{ina:este}

\paragraph{Veľmi dôležitá poznámka.}
Niekedy je potrebné nadpisom označiť odsek. Text pokračuje hneď za nadpisom.



\section{Dôležitá časť} \label{dolezita}




\section{Ešte dôležitejšia časť} \label{dolezitejsia}




\section{Záver} \label{zaver} % prípadne iný variant názvu



%\acknowledgement{Ak niekomu chcete poďakovať\ldots}


% týmto sa generuje zoznam literatúry z obsahu súboru literatura.bib podľa toho, na čo sa v článku odkazujete
\bibliography{literatura}
\bibliographystyle{plain} % prípadne alpha, abbrv alebo hociktorý iný
\end{document}
